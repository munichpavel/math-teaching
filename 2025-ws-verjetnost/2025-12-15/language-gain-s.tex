\subsection*{A solution}

\begin{enumerate}
\item For an arbitrary number of levels, $N$,

\begin{align*}
1 &= c \sum_{k=1}^N \left(\sum_{\ell=k}^N (2 \ell - k) + \sum_{\ell=1}^{k - 1} 1 \right)
\end{align*}

Let's fix $k$ and consider the inside summation,

\begin{equation*}
\sum_{\ell=k}^N (2 \ell - k) + \sum_{\ell=1}^{k - 1} 1
\end{equation*}

We call the remain-improve sub-sum for indices $\ell \geq k$ on the left of the $+$ sign, and the drop sub-sum when $\ell < k$ on the right. The drop sub-sum is simply counting the $k-1$ terms where the value is constant:
\begin{equation*}
\sum_{\ell=1}^{k - 1} 1 = k - 1
\end{equation*}

For the remain-improve sub-sum, we can simplify the calculation by splitting the summation:

\begin{align*}
\sum_{\ell=k}^N (2 \ell - k) &= 2 \sum_{\ell=k}^N \ell - \sum_{\ell=k}^N k \\
&= 2 \left( \frac{(k+N)(N-k+1)}{2} \right) - k(N-k+1)
\end{align*}

Factoring out the common term $(N-k+1)$:

\begin{align*}
&= (N-k+1) [ (k+N) - k ] \\
&= (N-k+1) N \\
&= N^2 + N - Nk
\end{align*}

Combining the two parts (still for fixed $k$), the total inner sum is:

\begin{align*}
(N^2 + N - Nk) + (k - 1) &= N^2 + N - 1 - k(N - 1)
\end{align*}

We put this result back into the outer summation over $k$:

\begin{align*}
1 &= c \sum_{k=1}^N \left( (N^2 + N - 1) - k(N - 1) \right)
\end{align*}

Evaluate the sum by splitting the constant term and the $k$-dependent term:

\begin{align*}
1 &= c \left[ \sum_{k=1}^N (N^2 + N - 1) - (N - 1) \sum_{k=1}^N k \right] \\
&= c \left[ N(N^2 + N - 1) - (N - 1) \frac{N(N+1)}{2} \right]
\end{align*}

Factor out $\frac{N}{2}$:

\begin{align*}
1 &= \frac{cN}{2} \left[ 2(N^2 + N - 1) - (N - 1)(N + 1) \right] \\
&= \frac{cN}{2} \left[ 2N^2 + 2N - 2 - (N^2 - 1) \right] \\
&= \frac{cN}{2} (N^2 + 2N - 1)
\end{align*}

Solving for $c$:

\begin{equation*}
c = \frac{2}{N(N^2 + 2N - 1)}
\end{equation*}

\item We calculate the conditional probability using the definition $P(Y=k+1 \mid X=k) = \frac{P(X=k, Y=k+1)}{P(X=k)}$.

\textbf{Numerator:}
Since $k+1 \geq k$, we use the first case of the joint distribution function with $\ell = k+1$:
\begin{equation*}
    P(X=k, Y=k+1) = c \cdot (2(k+1) - k) = c(2k + 2 - k) = c(k+2)
\end{equation*}

\textbf{Denominator:}
The marginal probability $P(X=k)$ is the sum of the joint probabilities over all possible values of $Y$. We split the sum into the cases $\ell < k$ and $\ell \geq k$:
\begin{align*}
    P(X=k) &= \sum_{\ell=1}^{N} P(X=k, Y=\ell) \\
    &= \sum_{\ell=1}^{k-1} c + \sum_{\ell=k}^{N} c(2\ell - k)
\end{align*}
Evaluating the sums (where the second sum has $N-k+1$ terms):
\begin{align*}
    P(X=k) &= c(k-1) + c\left[ 2\sum_{\ell=k}^{N} \ell - \sum_{\ell=k}^{N} k \right] \\
    &= c(k-1) + c\left[ (N-k+1)(N+k) - k(N-k+1) \right] \\
    &= c(k-1) + c\left[ (N-k+1)(N+k-k) \right] \\
    &= c(k-1) + cN(N-k+1) \\
    &= c(N^2 - Nk + N + k - 1)
\end{align*}

\textbf{Result:}
Substituting the numerator and denominator back into the conditional probability formula (the constant $c$ cancels out):
\begin{equation*}
    P(Y=k+1 \mid X=k) = \frac{k+2}{N^2 - Nk + N + k - 1}
\end{equation*}

\end{enumerate}