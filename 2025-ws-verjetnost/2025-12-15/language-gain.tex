\section{Exercise: language gains (discrete multivariable, marginal, independence)}

Assume we are given data from several years of participants in an intensive language program.
At the beginning of each course, students are assessed for their starting level, and given a proficiency score of Novice, Intermediate, Advanced or Superior.
At the end, they are also assessed. Per participant, the initial and final assessments are recorded.

Let $X$ be the random variable corresponding to a randomly chosen student's start level.
$X$ takes values $\{1, 2, 3, 4\}$, defined in bijection with the 4 ordered proficiency levels.

Similarly, let $Y$ be the random variable corresponding to a randomly chosen student's level at the end of the program, again taking values in $\{1, \ldots, 4\}$.

Assume we have estimated from the data that the probability distribution function of $(X, Y)$ is

\begin{equation*}
P(X=k, Y=\ell) =
    \begin{cases}
        c \cdot (2\ell - k) \quad \text{ if }\ell \geq k \\
        c & \quad \text{ otherwise,}
    \end{cases}
\end{equation*}
for some $c \in \mathbb{R}$.

\begin{enumerate}
\item Calculate the constant $c$.
\item Now consider an arbitrary, finite number of levels, $N \in \mathbb{N}$. Calculate $c$ in this case.
\item Marginal Distribution: Derive the marginal PMF for the Start Level, $p_X(x)$, for all $x \in \{1, 2, 3, 4\}$.
\item Independence: Are the random variables $X$ and $Y$ independent? Justify your answer using the definition of independence for specific coordinates (e.g., compare $p_{X,Y}(4,1)$ with $p_X(4)p_Y(1)$).
\item Probability of Regression: Calculate $P(Y < X)$. This represents the total probability that a student finishes at a lower level than they started.
\item Conditional probability
\end{enumerate}
