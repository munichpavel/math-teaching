\section{Exercise: Language gains, continuous model}

We now model the proficiency scores as continuous random variables on the unit square. Let $X$ (Start Score) and $Y$ (End Score) be random variables with support $0 \le x, y \le 1$.

We adopt a probability density function (PDF) that is weighted towards growth ($y \ge x$) but allows for non-zero probability of regression ($y < x$).
The density decreases linearly as the combined score $(x+y)$ increases, reflecting the difficulty of maintaining high growth at high levels, which matches both the ACTFL pyramid infographic and empirical data.

Let $(X, Y)$ be the multivariable distribution with density function
\begin{equation*}
f_{X,Y}(x,y) =
\begin{cases}
    c(2 - x - y) & \text{if } 0 \le x \le y \le 1 \\
    \frac{c}{2} & \text{if } 0 \le y < x \le 1 \\
    0 & \text{otherwise}
\end{cases}
\end{equation*}

\begin{enumerate}
    \item  Calculate the value of the normalization constant $c$ such that $f_{X,Y}(x,y)$.

    \item Let $E$ be the event that a student improves their score ($Y > X$). Calculate the conditional probability of improvement given a specific start score $X=x$:
    \begin{equation*}
    g(x) = P(Y > X \mid X = x)
    \end{equation*}

    \item Show that the function $g(x)$ derived above is strictly decreasing for $x \in [0, 1]$. Interpret this result in the context of the language program.
\end{enumerate}

