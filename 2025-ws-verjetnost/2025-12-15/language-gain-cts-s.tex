\subsection*{Solutions}

\begin{enumerate}

\item To find the constant $c$, we integrate the joint density function $f_{X,Y}(x,y)$ over the unit square $[0,1] \times [0,1]$. We split the integral into two regions based on the piecewise definition: the lower triangle ($y < x$) and the upper triangle ($y \geq x$).

\begin{align*}
    1 &= \iint_{S} f_{X,Y}(x,y) \, dy \, dx \\
    &= \int_0^1 \int_0^x \frac{c}{2} \, dy \, dx + \int_0^1 \int_x^1 c(2 - x - y) \, dy \, dx
\end{align*}

\begin{figure}
    \begin{center}
    \includegraphics[width=0.8\linewidth]{cts-language-gain-integration-region}
    \end{center}
  \caption{Integration region for piecewise $f_{X,Y}(x,y)$}
  \label{fig:language-gain-cts}
\end{figure}

Calculating the first integral (lower triangle):
\begin{equation*}
    L = c\int_0^1 \left(x/2 - 0 \right) dx = \frac{cx^2}{4} \bigg\rvert_0^1 = \frac{c}{4}
\end{equation*}

Calculating the second integral (upper triangle):
\begin{align*}
    R &= c \int_0^1 \left( 2y - xy - \frac{y^2}{2} \right) \bigg\rvert_{y=x}^1 dx \\
    &= c \int_0^1 \left( 2 - x - 1/2 - 2x + x^2 + x^2 / 2\right) dx \\
    &= c \int_0^1 \left( \frac{3x^2}{2}  - 3x + \frac{3}{2} \right) dx \\
    &= c \left( \frac{x^3}{2} - \frac{3x^2}{2} + \frac{3x}{2}\right) \bigg\rvert_0^1 \\
    & = \frac{c}{2}
\end{align*}


Summing them up:
\begin{equation*}
    1 = L + R = \frac{c}{4} + \frac{c}{2} = \frac{3c}{4} \implies c = \frac{4}{3}
\end{equation*}

\item We calculate the conditional probability $g(x) = P(Y > X \mid X = x)$ using the ratio of the conditional density mass where $y > x$ to the total marginal density at $x$.

We calculate the conditional probability $g(x) = P(Y > X \mid X = x)$. By the definition of conditional probability for continuous variables, this is the ratio of the conditional density where the event holds ($Y > X$) to the total marginal density at $X=x$.

The numerator represents the probability density accumulated specifically in the region where the student improves (where $y > x$), given their start score is fixed at $x$. Mathematically, we integrate the joint PDF $f_{X,Y}(x,y)$ with respect to $y$ strictly over the interval $(x, 1]$.

In this range ($y \ge x$), the joint PDF is defined as $c(2 - x - y)$. Thus:
\begin{align*}
    \text{Numerator} &= \int_x^1 f_{X,Y}(x,y) \, dy \\
    &= \int_x^1 c(2 - x - y) \, dy \\
    &= c \frac{3}{2} (x^2 - 2x + 1) \text{ from our previous calculations}\\
    &= c\frac{3}{2}(1-x)^2
\end{align*}

The denominator is the marginal density $f_X(x)$, which represents the total probability density of starting at $x$, regardless of the outcome $Y$. This is the sum of the density where scores decline ($y < x$) and where scores improve ($y \ge x$).
\begin{align*}
    f_X(x) &= \underbrace{\int_0^x \frac{c}{2} \, dy}_{\text{Decline ($y<x$)}} + \underbrace{\int_x^1 c(2 - x - y) \, dy}_{\text{Improvement ($y \ge x$)}} \\
    &= \frac{cx}{2} + c \frac{3}{2}(1-x)^2
\end{align*}


Substituting these back into the conditional probability formula:
\begin{align*}
    g(x) &= \frac{\int_x^1 f_{X,Y}(x,y) \, dy}{f_X(x)} \\
    &= \frac{c \frac{3}{2}(1-x)^2}{\frac{cx}{2} + c \frac{3}{2}(1-x)^2} \\
    &=  \frac{3(1-x)^2}{x + 3(1-x)^2}
\end{align*}


\item
We analyze the monotonicity of $g(x)$ for $x \in [0, 1]$. It is easier to analyze the reciprocal function $h(x) = 1/g(x)$:
\begin{equation*}
    h(x) = \frac{x + 3(1-x)^2}{3(1-x)^2} = 1 + \frac{x}{3(1-x)^2}
\end{equation*}
Let $k(x) = \frac{x}{(1-x)^2}$. Taking the derivative with respect to $x$:
\begin{align*}
    k'(x) &= \frac{1 \cdot (1-x)^2 - x \cdot 2(1-x)(-1)}{(1-x)^4} \\
    &= \frac{(1-x) + 2x}{(1-x)^3} \\
    &= \frac{1+x}{(1-x)^3}
\end{align*}
For $x \in [0, 1)$, $k'(x)$ is strictly positive, meaning $h(x)$ is strictly increasing. Therefore, $g(x)$ is \textit{strictly decreasing}.

The function $g(x)$ represents the probability that a student improves their score given their starting level. The fact that $g(x)$ is strictly decreasing implies that beginners are much more likely to show improvement than advanced students.

This continuous model aligns more closely with the inverted ACTFL language gain pyramid we saw before:

\begin{figure}
    \begin{center}
    \includegraphics[width=0.8\linewidth]{actfl-pyramid}
    \end{center}
  \caption{Infographic depicting inverted language gain pyramid from \href{https://www.languagetesting.com/actfl-proficiency-scale}{https://www.languagetesting.com/actfl-proficiency-scale}}
  \label{fig:actfl-pyramid}
\end{figure}

\end{enumerate}
