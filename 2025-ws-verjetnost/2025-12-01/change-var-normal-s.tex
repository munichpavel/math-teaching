\subsection*{A solution}

Note: as mentioned during tutorials, I have a tendency to be verbose. This level of explanatory detail is likely not required for full marks on an exam answer.

This problem can be solved by finding an appropriate change of variable, and then using this change of variable correctly to convert the integral corresponding to (the definition of) $F_X(x)$ as $F_X(x) = \int_{-\infty}^{x} f_X(s) ds$ into an integral that involves the probability distribution function of the standard normal random variable.
If you are already comfortable with the ins and outs of integrating combined with change of variables,  then feel free to skip head to Section \ref{sec:int-solution-only}.

\subsubsection{Solution background}\label{sec:int-solution-background}

In general, if $\phi: [a, b] \subseteq \mathbb{R} \to [\phi(a) , \phi(b)] \subseteq \mathbb{R}$ is a continuous, 1x differentiable function with a well-defined, at least once differentiable inverse $\phi^{-1}$, then the following integral $I$ can be written under the change of variable $\phi$ as

\begin{equation}\label{eq:int-change-var}
I = \int_{a}^{b} f(s) ds = \int_{\phi(a)}^{\phi(b)} f(\phi^{-1}(t)) \left( \frac{d \phi^{-1}}{dt} \right) dt
\end{equation}

\noindent In even greater generality, the above only need hold off a set of Lebesgue measure 0 in $[a, b]$.

Note: Equation (\ref{eq:int-change-var}) assumes we have a \textbf{signed integral}. For the original integral we have chosen $a < b$.
For the transformed version under $\phi$, it need not be the case that $\phi(a') \leq \phi(b')$ for all $a \leq a' < b \leq b'$.
For calculations, we need to break the interval $[\phi(a), \phi(b)]$ (or $[\phi(b), \phi(a)]$ if $\phi(a) \geq \phi(b)$) into subintervals $[a', b']$ where $\phi$ is either strictly non-increasing and non-decreasing, and then adjust the sign accordingly with a multiple of -1 if $\phi(a') > \phi(b')$.

Now we return to our specific example, with

\begin{align*}
I &= I(x) = \int_{-\infty}^{x} f_X(s) ds \\
&= \int_{0}^x \frac{c}{s^{3/2}} \exp\left(\frac{-1}{2s}\right) ds, &\text{as $f_X(s) = 0$ for $s \leq 0$,}
\end{align*}

\subsubsection{A solutions sans background}\label{sec:int-solution-only}
Since the probability distribution of $Z$ is $f_Z = \frac{\exp(-z^2)}{\sqrt{2 \pi}}$, we want a transformation whose inverse satisfies, for $t > 0$, $\frac{-1}{2 \phi^{-1}(t)} = -\frac{t^2}{2}$.
Let's consider then
\begin{equation*}
(s = )\,\phi^{-1}(t) = \frac{1}{t^2}
\end{equation*}
with first derivative

\begin{align*}
\frac{d}{dt} \phi^{-1}(t) &= - \frac{1}{t^3}
\end{align*}


This function is continuous (and differentiable) on $(0, \infty)$. It is strictly monotonically decreasing there as well, so it is invertible, and has inverse

\begin{equation*}
(t = )\,\phi(s) =  \frac{1}{\sqrt{s}}
\end{equation*}
which is also at least once-differentiable on $(0, \infty)$.

Since $\lim_{s \rightarrow 0^{+}} \phi(s) = +\infty$, we write the integral using the change of variable $\phi$ now as


\begin{align*}
I(x) &= c \int_{\infty}^{\frac{1}{\sqrt{x}}}
\frac{1}{ \frac{1}{t^3}} \exp\left( \frac{-t^2}{2}\right) \left( -  \frac{1}{t^3} \right) dt \\
&= -c \int_{\infty}^{\frac{1}{\sqrt{x}}} \exp\left( \frac{-t^2}{2}\right) dt \\
&= c \int_{\frac{1}{\sqrt{x}}}^{\infty} \exp\left( \frac{-t^2}{2}\right) dt \\
& = c \sqrt{2 \pi} \int_{\frac{1}{\sqrt{x}}}^{\infty} \frac{1}{\sqrt{2 \pi}} \exp\left( \frac{-t^2}{2}\right) dt \\
& = c \sqrt{2 \pi} P\left(Z \geq \frac{1}{\sqrt{x}} \right)
\end{align*}
where the final equality holds due to the definition of $F_Z$ as

\begin{equation*}
F_Z(z) = P(Z \leq z) =  \int_{-\infty}^x 1/\sqrt{2 \pi} \exp^{-1 / 2 t^2} df,
\end{equation*}
and using that $P(Z > z) = 1 - F_Z(z)$ together with $P(Z=z) = 0$ (set of measure zero).


% \begin{align*}
% I(x) &= c \sqrt{2 \pi} \left(1 - \int_{-\infty}^{\frac{1}{\sqrt{x}}} \frac{1}{\sqrt{2 \pi}} \exp\left( \frac{-t^2}{2}\right) dt \right) \\
% &= c \sqrt{2 \pi} \left( 1 - \Phi(\frac{1}{\sqrt{x}}) \right) \\
% &= c \sqrt{2 \pi} P\left(Z > \frac{1}{\sqrt{x}}\right) \\

% \end{align*}

% \begin{align*}
% I(x) &= c \int_{\infty}^{\left( \frac{\pi}{2} \right)^{1/4} \frac{1}{\sqrt{x}}} \frac{1}{\left( \frac{\pi}{2} \right)^{3/4} \frac{1}{t^3}} \exp\left( \frac{-1}{2 \sqrt{\frac{\pi}{2}} \frac{1}{t^2}}\right)\left( - \sqrt{2 \pi} \frac{1}{t^3} \right) dt \\
% &= -c \int_{\infty}^{\left( \frac{\pi}{2} \right)^{1/4} \frac{1}{\sqrt{x}}} \\
% \end{align*}

% \emph{Notation aside}: You might justifiably ask why I started with $\phi$
% since its inverse is $\phi^{-1}(t) = 1/t^2$, which, when substituted into the exponential, would give us a term $\exp\left(-1/ t^2\right)$ like with the probability density function of a normal distribution.


% and $\phi(s) = 1/s^2$. On the domain $(0, \infty)$, we are in the simpler situation of $\phi$ being strictly monotonically increasing and differentiable (actually, infinitely differentiable or \emph{smooth}, though we don't need that property here), so $\phi^{-1}(t)$ exists and is differentiable on the range of $\phi$, which also happens to be $(0, \infty)$.

% x Hence we don't need to worry about premultiplying with a factor of $-1$ to account for the signed integral of Equation (\ref{eq:int-change-var}).
% Furthermore, $f_X$ is a non-negative function by its definition as a probability density function (or equivalently, measure function).



% \begin{align*}
% I(x) &= c \int_{0}^{1/x^2} \frac{1}{} ds \\
% &= \int_{0}^x \frac{c}{s^{3/2}} \exp\left(\frac{-1}{2s}\right) ds, &\text{as $f_X(s) = 0$ for $s \leq 0$,}
% \end{align*}