\documentclass[12pt]{article}
\usepackage{amsmath, amssymb, amsthm}
\usepackage{enumerate}
\usepackage[T1]{fontenc}
\usepackage[utf8]{inputenc}
\usepackage[english]{babel}
\usepackage{geometry}
\geometry{
    a4paper,
    margin=4cm,
}
\usepackage{setspace}


\title{Probability Exercises, 2025-12-01}

\begin{document}
\maketitle
\tableofcontents

\onehalfspacing % Set line spacing to 1.5

\subsection{Exercise, hat (or jacket check) problem}

$n$ visitors to a museum in Berlin all check in black, Jack Wolfskin jackets upon arrival before 6pm. When they come to pick up their jackets later in the evening around 8pm, with uniform probability, the coat check clerk gives them a random black, Jack Wolfskin jacket.

Let $X$ be the random variable defined as the number of visitors who get the correct jacket back, so $X \in \{0, \ldots, n\}$.

On average, how many visitors do we expect to get back their own jacket?

\smallskip

\emph{Hint:} Use indicator functions / variables, and expected value.

\subsubsection*{A solution}

Let $J_k$ be the event that the $k^{\text{th}}$ visitor gets their jacket back. By assumption of uniformity, $P(J_k) = 1/n$.

Define the indicator function $I_k$ as
\[
I_k = 1 \text{ if } A_k \text{ occurs; else } 0
\]
In other words, $I_k$ is the indicator function for the $k^{\text{th}}$ person getting their correct jacket back.

Since $X = I_1 + \ldots I_n$, we can use linearity of expectation to calculate
\[
E(X) = E\left(\sum_{j=1^n} I_j\right) = \sum_{j=1^n} E(I_j)
\]
but $E(I_j) = 1 \cdot 1/n + 0 \cdot (n-1) / n = 1/n$ for each $j$, so
\[
E(X) = n \cdot 1/n = 1
\]
Hence we expect one person to get their correct jacket back.

\emph{Mathematical aside:} Returning to the question about the police-officer and hooligan configurations, in this exercise, the trials in question (picking up your jacket) are \emph{distinguishable}, as the setup requires that the similar looking jackets be identifiable as belonging to one owner.


\section{Exercise (expected value, method of indicators): A variant on the blue and red pills}

(Adapted from Stirzaker, Probability and Random Variables, Chapter 5, ex. 11)

In the movie the Matrix, the character Morpheus offers another character Neo two pills. If Neo chooses the blue pill, he will return to his old life. If he chooses the red one, his eyes will be opened to some truth, and he will join Morpheus' band.

In this variant (also of Pólyeva's urn, see MR, vaja 4), Neo has to choose blindly from a bucket initially containing one red and one blue pill. If he picks red, he must swallow it and the game is over. If he picks blue, then he returns the blue pill and Morpheus adds an extra red pill.


Let $X$ be the number of draws. Calculate

\begin{enumerate}
\item The distribution $P(X>k)$
\item $\text{E}(X)$ (hint: use the method of indicators)
\end{enumerate}

\subsection*{A solution for 1, $P(X=k)$}

This is a geometric like process (a number of successive failures followed by a single success), except that the probability of success or failure changes each time.


Now suppose $k \in \{1, \ldots, 4\}$. By Morphius' rules, ${X > k}$ means that the first $k$ draws were blue

\begin{align*}
P(X > k) &= \prod_{j=1}^k (j+1) = \frac{1}{(k+1)!}
\end{align*}



\subsection*{A solution with indicator functions for 2, $\text{E}(X)$}

Let $I_j$ be the event that Neo draws again at step $j$. By construction, $X =\sum_{j=1}^\infty I_j$, and by linearity of expectation,

\begin{align*}
\text{E}(X) &= \sum_{j=1}^\infty E(I_j) \\
&=  \sum_{j=1}^\infty P(X \geq j) \\
&= \sum_{j=1}^\infty P(X > j-1) = \sum_{j=1}^\infty \frac{1}{j!}
\end{align*}

But $\sum_{j=1}^\infty \frac{1}{j!} = e - 1 \sim 1.718$, so $\text{E}(X) = e - 1$.


\section{Exercise (integral, transformation of continuous random variable)}

Let the continuous random variable $X$ be defined as having distribution


\begin{equation*}
f_X(x)=\left \{
\begin{array}{ll}
\frac{c}{x^{3/2}} \exp\left(\frac{-1}{2x}\right) & \mbox{for $x > 0$} \\
0 & \mbox{otherwise} \\
\end{array}
\right .
\end{equation*}

where $c$ is some constant chosen so that $\int_{-\infty}^{\infty}f_X(x) = 1$.

Part 1: Show that, if $Z \sim N(0,1)$, for $x > 0$ we have

\begin{equation*}
F_X(x) = 2 P_Z\left(Z \geq \frac{1}{\sqrt{x}}\right).
\end{equation*}

\subsection*{A solution}

Note: as mentioned during tutorials, I have a tendency to be verbose. This level of explanatory detail is likely not required for full marks on an exam answer.

This problem can be solved by finding an appropriate change of variable, and then using this change of variable correctly to convert the integral corresponding to (the definition of) $F_X(x)$ as $F_X(x) = \int_{-\infty}^{x} f_X(s) ds$ into an integral that involves the probability distribution function of the standard normal random variable.
If you are already comfortable with the ins and outs of integrating combined with change of variables,  then feel free to skip head to Section \ref{sec:int-solution-only}.

\subsubsection{Solution background}\label{sec:int-solution-background}

In general, if $\phi: [a, b] \subseteq \mathbb{R} \to [\phi(a) , \phi(b)] \subseteq \mathbb{R}$ is a continuous, 1x differentiable function with a well-defined, at least once differentiable inverse $\phi^{-1}$, then the following integral $I$ can be written under the change of variable $\phi$ as

\begin{equation}\label{eq:int-change-var}
I = \int_{a}^{b} f(s) ds = \int_{\phi(a)}^{\phi(b)} f(\phi^{-1}(t)) \left( \frac{d \phi^{-1}}{dt} \right) dt
\end{equation}

\noindent In even greater generality, the above only need hold off a set of Lebesgue measure 0 in $[a, b]$.

Note: Equation (\ref{eq:int-change-var}) assumes we have a \textbf{signed integral}. For the original integral we have chosen $a < b$.
For the transformed version under $\phi$, it need not be the case that $\phi(a') \leq \phi(b')$ for all $a \leq a' < b \leq b'$.
For calculations, we need to break the interval $[\phi(a), \phi(b)]$ (or $[\phi(b), \phi(a)]$ if $\phi(a) \geq \phi(b)$) into subintervals $[a', b']$ where $\phi$ is either strictly non-increasing and non-decreasing, and then adjust the sign accordingly with a multiple of -1 if $\phi(a') > \phi(b')$.

Now we return to our specific example, with

\begin{align*}
I &= I(x) = \int_{-\infty}^{x} f_X(s) ds \\
&= \int_{0}^x \frac{c}{s^{3/2}} \exp\left(\frac{-1}{2s}\right) ds, &\text{as $f_X(s) = 0$ for $s \leq 0$,}
\end{align*}

\subsubsection{A solutions sans background}\label{sec:int-solution-only}
Since the probability distribution of $Z$ is $f_Z = \frac{\exp(-z^2)}{\sqrt{2 \pi}}$, we want a transformation whose inverse satisfies, for $t > 0$, $\frac{-1}{2 \phi^{-1}(t)} = -\frac{t^2}{2}$.
Let's consider then
\begin{equation*}
(s = )\,\phi^{-1}(t) = \frac{1}{t^2}
\end{equation*}
with first derivative

\begin{align*}
\frac{d}{dt} \phi^{-1}(t) &= - \frac{1}{t^3}
\end{align*}


This function is continuous (and differentiable) on $(0, \infty)$. It is strictly monotonically decreasing there as well, so it is invertible, and has inverse

\begin{equation*}
(t = )\,\phi(s) =  \frac{1}{\sqrt{s}}
\end{equation*}
which is also at least once-differentiable on $(0, \infty)$.

Since $\lim_{s \rightarrow 0^{+}} \phi(s) = +\infty$, we write the integral using the change of variable $\phi$ now as


\begin{align*}
I(x) &= c \int_{\infty}^{\frac{1}{\sqrt{x}}}
\frac{1}{ \frac{1}{t^3}} \exp\left( \frac{-t^2}{2}\right) \left( -  \frac{1}{t^3} \right) dt \\
&= -c \int_{\infty}^{\frac{1}{\sqrt{x}}} \exp\left( \frac{-t^2}{2}\right) dt \\
&= c \int_{\frac{1}{\sqrt{x}}}^{\infty} \exp\left( \frac{-t^2}{2}\right) dt \\
& = c \sqrt{2 \pi} \int_{\frac{1}{\sqrt{x}}}^{\infty} \frac{1}{\sqrt{2 \pi}} \exp\left( \frac{-t^2}{2}\right) dt \\
& = c \sqrt{2 \pi} P\left(Z \geq \frac{1}{\sqrt{x}} \right)
\end{align*}
where the final equality holds due to the definition of $F_Z$ as

\begin{equation*}
F_Z(z) = P(Z \leq z) =  \int_{-\infty}^x 1/\sqrt{2 \pi} \exp^{-1 / 2 t^2} df,
\end{equation*}
and using that $P(Z > z) = 1 - F_Z(z)$ together with $P(Z=z) = 0$ (set of measure zero).


% \begin{align*}
% I(x) &= c \sqrt{2 \pi} \left(1 - \int_{-\infty}^{\frac{1}{\sqrt{x}}} \frac{1}{\sqrt{2 \pi}} \exp\left( \frac{-t^2}{2}\right) dt \right) \\
% &= c \sqrt{2 \pi} \left( 1 - \Phi(\frac{1}{\sqrt{x}}) \right) \\
% &= c \sqrt{2 \pi} P\left(Z > \frac{1}{\sqrt{x}}\right) \\

% \end{align*}

% \begin{align*}
% I(x) &= c \int_{\infty}^{\left( \frac{\pi}{2} \right)^{1/4} \frac{1}{\sqrt{x}}} \frac{1}{\left( \frac{\pi}{2} \right)^{3/4} \frac{1}{t^3}} \exp\left( \frac{-1}{2 \sqrt{\frac{\pi}{2}} \frac{1}{t^2}}\right)\left( - \sqrt{2 \pi} \frac{1}{t^3} \right) dt \\
% &= -c \int_{\infty}^{\left( \frac{\pi}{2} \right)^{1/4} \frac{1}{\sqrt{x}}} \\
% \end{align*}

% \emph{Notation aside}: You might justifiably ask why I started with $\phi$
% since its inverse is $\phi^{-1}(t) = 1/t^2$, which, when substituted into the exponential, would give us a term $\exp\left(-1/ t^2\right)$ like with the probability density function of a normal distribution.


% and $\phi(s) = 1/s^2$. On the domain $(0, \infty)$, we are in the simpler situation of $\phi$ being strictly monotonically increasing and differentiable (actually, infinitely differentiable or \emph{smooth}, though we don't need that property here), so $\phi^{-1}(t)$ exists and is differentiable on the range of $\phi$, which also happens to be $(0, \infty)$.

% x Hence we don't need to worry about premultiplying with a factor of $-1$ to account for the signed integral of Equation (\ref{eq:int-change-var}).
% Furthermore, $f_X$ is a non-negative function by its definition as a probability density function (or equivalently, measure function).



% \begin{align*}
% I(x) &= c \int_{0}^{1/x^2} \frac{1}{} ds \\
% &= \int_{0}^x \frac{c}{s^{3/2}} \exp\left(\frac{-1}{2s}\right) ds, &\text{as $f_X(s) = 0$ for $s \leq 0$,}
% \end{align*}

\subsection{Exercise (Transformation of random variable): uniform}

Adapted from Stirzaker, Probability and Random Variables, Example 5.5.3

Let $X \sim U(a, b)$, $0 < a <  b < 1\subseteq \mathbb{R}$, and let $Y = -\frac{1}{\lambda} \log X$, where $\lambda > 0$ and $\log$ is the natural logarithm.

Determine $F_Y(y)$.

\subsubsection*{A solution}

By definition of $Y$,

\begin{align*}
F_Y(y) &= P_Y(Y \leq y) \\
&= P_X( -\frac{1}{\lambda} \log X \leq y) \\
&= P_X(\log X \geq -\lambda y) \text{, since $\lambda > 0$} \\
&= P_X(X \geq \exp(-\lambda y)) \text{, since $\exp$ preserves ordering on $\mathbb{R}$}
\end{align*}

The probability density function of $U(a, b)$ is $\frac{1}{b-a} \cdot I_{s \in[a, b]}$, where here we mean $I$ as the indicator function ($=1$ for $s \in [a, b]$, otherwise 0).
Thus we have

\begin{align*}
F_Y(y) &= \int_{\exp(-\lambda y)}^{\infty} \frac{1}{b-a} \cdot I_{s \in[a, b]} ds \\
&= \frac{1}{b - a} \int_{\exp(-\lambda y)}^{\infty} I_{s \in[a, b]} ds
\end{align*}

We consider now cases on $y$ values.
If $\exp(-\lambda y) > b$, we have $F_Y(y) =0 $. This condition is equivalent to
\begin{align*}
    &-\lambda y > \log(b), \text { since $\log$ preserves ordering on $\mathbb{R}$, and} \\
    &y < - \frac{1}{\lambda} \log(b), \text{ since $\lambda > 0$.}
\end{align*}

Next, if $ - \frac{1}{\lambda} \log(b) \leq y \leq - \frac{1}{\lambda} \log(a)$, we have $F_Y(y) = \frac{b - \exp(-\lambda y)}{b - a}$.

Finally, if $y > - \frac{1}{\lambda} \log(a)$, then the integral evaluates to $b - a$, and $F_Y(y) = 1$ for such $y$ values.


Putting the pieces together,

\begin{equation}
F_Y(y)=
    \begin{cases}
        0 & \text{if } y < -\frac{1}{\lambda} \log(b) \\
        \frac{b - \exp(-\lambda y)}{b - a} & \text{if }- \frac{1}{\lambda} \log(b) \leq y \leq - \frac{1}{\lambda} \log(a) \\
        1 & \text {if } y > - \frac{1}{\lambda} \log(a).
    \end{cases}
\end{equation}

The resulting  $Y$ is called a truncated exponential distribution.


% \subsubsection*{A solution}

% By definition of $Y$,

% \begin{align*}
% F_Y(y) &= P_Y(Y \leq y) \\
% &= P_X( -\frac{1}{\lambda} \log X \leq y) \\
% &= P_X(\log X \geq -\lambda y) \text{, since $\lambda > 0$} \\
% &= P_X(X \geq \exp(-\lambda y)) \text{, since $\exp$ preserves ordering on $\mathbb{R}$} \\
% &= 1 - P(X \leq \exp(-\lambda y))
% \end{align*}

% The probability density function of $U(a, b)$ is $\frac{s-a}{b-a} \cdot I_{s \in[a, b]}$, where here we mean $I$ as the indicator function ($=1$ for $s \in [a, b]$, otherwise 0).
% Thus we have

% \begin{align*}
% F_Y(y) &= 1 - \int_{-\infty}^{\exp(-\lambda y)} \frac{s-a}{b-a} \cdot I_{s \in[a, b]} ds
% \end{align*}

% Since $0 \leq a < b \leq 1$, and $I_{s \in[a, b]}$ is 0 outside $[a, b]$, we'll define $F_Y$ piecewise depending on the values of $y$.

% If the upper integral limit satisfies

% \begin{align*}
%     \exp(-\lambda y) &< a, \text{ which implies } \\
%     -\lambda y &< \log(a), \text { since $\log$ preserves ordering on $\mathbb{R}$, giving} \\
%     y &> - \frac{1}{\lambda} \log(a), \text{ since $\lambda < 0$,}
% \end{align*}
% then $F_Y(y) = 0$, because the indicator function is identically 0 across the range of integration.

% Next consider $a \leq \exp(-\lambda y) \leq b$.
% As above, this implies

% \begin{equation*}
% - \frac{1}{\lambda} \log(b) \leq y \leq - \frac{1}{\lambda} \log(a),
% \end{equation*}

% and for such $y$ we have

% \begin{align*}
% F_Y(y) &= 1 - \int_{a}^{\exp(-\lambda y)} \frac{s-a}{b-a} \cdot I_{s \in[a, b]} ds \\
% &= 1 - \frac{\exp(-\lambda y) - a}{b - a} \\
% &= \frac{b - \exp(-\lambda y)}{b - a}
% \end{align*}

% Finally, for $y < -\frac{1}{\lambda} \log(b)$, $F_Y(y) = 1$.
% Putting the pieces together,

% \begin{equation}
% F_Y(y)=
%     \begin{cases}
%         1 & \text{if } y < -\frac{1}{\lambda} \log(b) \\
%         \frac{b - \exp(-\lambda y)}{b - a} & \text{if }  - \frac{1}{\lambda} \log(b) \leq y \leq - \frac{1}{\lambda} \log(a) \\
%         0 & \text{if } y > - \frac{1}{\lambda} \log(a)
%     \end{cases}
% \end{equation}

% \begin{equation}
% \chi_{\mathbb{Q}}(x)=
%     \begin{cases}
%         1 & \text{if } x \in \mathbb{Q}\\
%         0 & \text{if } x \in \mathbb{R}\setminus\mathbb{Q}
%     \end{cases}
% \end{equation}



\end{document}